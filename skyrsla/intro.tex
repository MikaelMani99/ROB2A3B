\section{Inngangur}
Vélmenið sem við ætlum að gera á þessari önn verður raspberry pi vélmenni sem er stýrt með vefappi í gegnum síma. Mestmegnis af kóðanum sem vélmennið keyrir verður kóðað með python. Hinsvegar verður kóðinn á bak við vefappið mest megnis í JavaScript. Vélmennið ætti að vera með myndavél og streama live feed til notandans sem er að stjórna vélmenninu. Við ætlum að nota motora úr Vex kittinu og stjórna þeim með raspberry pi. Við ætlum að setja upp myndavél og flygja \cite{prasad2014smart} til að fá hana til að virka. Við notum raspberry pi vegna þess að það er auðvelt að nota og er 
Notgildi vélmenisins gæti verið margvíslegur, hann gæti verið til dæmis notaður sem eftirlits vélmenni sem þú gætir tengst þegar þú ert ekki heima hjá þér til að keyra um og athuga hvort það sé ekki allt á réttum stað heima hjá þér meðan þú ert í burtu. Einnig væri hægt að nota vélmennið til að fara á staði þar sem fólk kemmst ekki eins og í loftræsikerfi til sjá hvort það sé eitthvað fast þar.



Skoðaði code complete /cite{mcconnell2004code}
Svo er líka hægt að nota tilvitnanir í vef \cite{WinNT}
\begin{figure}[h]
\includegraphics[scale=.3]{img/system}
\end{figure}