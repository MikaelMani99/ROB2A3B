\section{Inngangur}
Vélmennið sem við ætlum að byggja og forrita á þessari önn verður vélmenni sem er hægt að stjórna í gegnum síma með vefappi, við ætlum að nota RaspberryPi sem gruninn af vélmenninu.Kóðinn sem vélmennið keyrir verðir að mestu leyti python kóði og það er tungumálið sem við völdum að nota til að kóða vélmennið en hinsvegar kóðinn á bakvið vefappið mun að mestu leyti vera JavaScript kóði. Vélmennið verður með Raspberry Pi myndavél sem mun vera notuð til að leyfa notendanum að sjá það sem vélmennið er að horfa á. Það mun stream-a live feed til vefappið þannig að stjórnandinn getur séð hvað hann er að gera um leið og hann er að keyra vélmennið. Við ætlum að nota motora úr Vex kittinu og stjórna þeim með Raspberry Pi. Við ætlum að setja upp myndavél og flygja \cite{prasad2014smart} til að fá hana til að virka. Á vefappinu mun notandinn geta séð það sem myndavélin er að stream-a og hann á líka að geta stjórnað vélmenninu með snerti stýringum á símanum sínum. Við ákvöðum að nota Raspberry Pi yfir hinum valmöguleikunum, Arduino og Vex, vegna þess að það er mjög einfalt og þægilegt að nota það. Raspberry Pi er með built-in Python þannig við getum notað það til að forrita vélmennið og vegna þess að Raspberry Pi-ið er sín eigin lítil tölva með net tengingu lætur það einfaldara fyrir okkur að ná að tengja það við vefappið þannig við getum stjórnað vélmenninu.

Við ákvöðum að nota Python til að forrita vélmennið útaf nokkrum einföldum ástæðum.Vegna þess að Python er mjög létt forritunartungumál að kóða í, það er léttar fyrir okkur að skrifa út dæmi og prófa hugmyndir okkar í Python meðað við önnur tungumál eins og til dæmis C. Raspberry Pi er mjög samhæft við python og það verður létt að nota Python til að forrita og láta það virka með Raspberry Pi-inu. Við ætlum að nota JavaScript fyrir vefappið vegna þess að JavaScript er létt og þægilegt tungumál að vinna með.

Svona tegund af vélmenni hefur mjög magvísilegt notgildi. Vélmennið gæti verið notað sem eftirlits vélmenni sem þú getur tengst þegar þú ert út úr húsi og vilt fylgjast með og athuga að allt sé eðlilegt og að allt sé á réttum stað heima hjá þér meðan þú ert í burtu. Hann getur líka verið notaður til að skoða staði sem fólk er of stórt til að komast inn léttilega eins og til dæmis loftræsikerfi. Þú gætir látt vélmennið inn í þar inn og keyrt um til að sjá ef það er eitthvað fast þar. Við gætum minnkað hversu stórt vélmennið er og látt það eins lítið og mögulega hægt er og svona mini vélmenni gætu verið notaður af sérsveit sem er að reyna bjarga húsi sem er búið að hertaka, vélmennið gæti verið sent inn á undan og fundið meira út um húsið og hvernig það er þar inni. 

Við byggðum vélmennið eins lítin og við gátum, notuðum stál grind til að búa till lítinn kassa sem við tengdum allt við sá grind. 2 Mótórar sem keyra 2 hjól hver og svo er Raspberry Pi-ið fastur á sínum eigin pall ofaná vélmenninu.

\begin{figure}[h]
\includegraphics[scale=.3]{img/system}
\end{figure}